%Interpret the observed connectivity changes in light of prior FUS neuromodulation literature. Address:
%\begin{itemize}
%    \item Biological mechanisms that could explain the modulation of sgACC-centered networks.
%    \item Clinical implications for treating mood disorders or pain.
%    \item Limitations such as sample size, targeting accuracy, or confounding physiological signals.
%    \item Future directions for closed-loop stimulation, multi-modal imaging, or larger cohorts.
%\end{itemize}
%Close with a concise statement that reinforces the manuscript's contribution to biomedical engineering.

\subsubsection*{Interpretation and rigor of the mixed-effects inference}

By moving to a subject-level mixed-effects framework, we obtain a conservative and transparent assessment of how tFUS modulates sgACC-centered connectivity over time. This model explicitly acknowledges (i) the within-subject structure of the design, (ii) the non-independence of individual sgACC--target edges, and (iii) the empirically observed baseline difference between active and sham sessions. Within this rigorous setup, the key finding is that sgACC connectivity shows a significantly larger increase from pre- to post-sonication in the active condition than in the sham condition, as captured by the positive $\beta_5$ interaction term. In other words, after accounting for each subject's baseline sgACC FC---and without assuming equal starting points---active tFUS is associated with a selective, sustained enhancement of sgACC functional coupling that is not explained by sham or generic temporal drift. The absence of a strong differential effect during stimulation and the modest sample size argue against overclaiming an immediate ``online'' effect in this dataset; instead, the pattern is more consistent with a delayed or accumulating influence of tFUS on sgACC network integration. Importantly, because this analysis treats subjects as the unit of inference and models differences-in-differences directly, it avoids inflated precision from edge-wise pseudoreplication. % and provides a statistically defensible foundation on which to build the extended mechanistic and network-level analyses for the full TBME manuscript.

\subsubsection*{Pooling tFUS and Post supports a stimulation-specific effect beyond RTM}
        
Pooling the two follow-up windows with a time-window fixed effect provides a single, well-powered estimate of the baseline--change slope per condition while adjusting for any overall Post--tFUS mean difference. Under a pure RTM account, both conditions should exhibit similarly negative slopes. Instead, we observe a robust negative slope in sham but a flat/positive slope in active, with a significant pooled slope difference. This pattern argues against a simple ``return to equilibrium'' explanation and supports the interpretation that active tFUS alters the baseline--change relationship itself, consistent with a stimulation-specific reconfiguration of sgACC-centered functional connectivity.


\subsubsection*{Network-level interpretation}

The network-level analysis extends this narrative by showing that the post-tFUS increase in sgACC connectivity is not distributed uniformly across the brain. Active stimulation preferentially boosts sgACC coupling with the frontoparietal Control network while slightly reducing its engagement with the Default Mode network (DMN) and modestly increasing coupling with Salience, Somatomotor, and Visual systems. The Control network is composed of flexible hubs in lateral prefrontal and parietal cortex that coordinate goal-directed behavior, working memory, and cognitive reappraisal \cite{yeo2011organization,cole2013multi}. Strengthening sgACC--Control interactions therefore aligns with a shift toward top-down regulation circuits that are often hypoactive in mood and affective disorders. In parallel, the small negative active--sham difference observed for the DMN ($\Delta$FC $=-0.039$ at post) is consistent with long-standing reports that sgACC hyperconnectivity with the DMN tracks maladaptive rumination and depressive symptom burden \cite{whitfield2012default}. Together, these patterns suggest that low-intensity tFUS nudges sgACC connectivity away from internally oriented DMN loops and toward executive-control hubs that are better positioned to stabilize affect.

Because adding the Visual network expands the multiple-comparison burden, the Control-network effects now fall just shy of the FDR threshold ($q=0.054$ for both windows), yet they remain the largest differences in the dataset and retain strong uncorrected support ($p<0.01$). Positive trends in Salience ($q_{\mathrm{FDR}}=0.17$), Somatomotor ($q_{\mathrm{FDR}}=0.13$), and Visual ($q_{\mathrm{FDR}}\approx 0.40$) circuits (Table~\ref{tab:network_specificity}) hint that tFUS may also recruit broader distributed networks that support interoceptive monitoring, motor readiness, and sensory reweighting. Replication in larger cohorts will be necessary to determine whether these secondary effects are reliable and whether they scale with behavioral outcomes. Nonetheless, the present results demonstrate that focal ultrasound neuromodulation can bias sgACC network embedding in a direction that is both mechanistically interpretable and clinically relevant.
